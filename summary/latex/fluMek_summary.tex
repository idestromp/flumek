\documentclass[10pt, A4]{article}
\usepackage[utf8]{inputenc}
\usepackage{lipsum}
\usepackage{comment} % environment for multiline comments. 
\usepackage{amsmath}
\usepackage{fancyhdr} 		
\usepackage{framed,color}
\definecolor{shadecolor}{rgb}{0.9,0.95,1}

% Use this package to modify the width of the pages
\usepackage[
    height=9in,		% height of the text block
    width=7in,		% width of the text block
    top=78pt,		% distance of the text block from the top of the page
    headheight=100pt,	% height for the header block
    headsep=12pt,	% distance from the header block to the text block
    heightrounded,	% ensure an integer number of lines
    %showframe,		% show the main blocks
    verbose,			% show the values of the parameters in the log file
    inner=3cm,		% width of inner margin
    outer=3cm		% width of outer margin
]{geometry}





\title{"insert course name here" summary}
\author{Petter Ideström}
\date{\today}



\begin{document}

\begin{titlepage}


\maketitle
\end{titlepage}

\tableofcontents

\newpage 

\section{stuff to repeat/learn }

\begin{itemize}
\item gauss law 
\item remember the volume element. 
\item divergence of 
\begin{equation}
\frac{\vec{\hat{r}}}{r^2}
\end{equation}
\end{itemize}


\section{Electrostatics}

\subsection{coulombs law}

\begin{shaded}
The force on a test charge $Q$ due to a single point charge $q$, that is at rest a distance $\vec{r_i}$ away is given by coulombs law. 
\begin{equation}
\vec{F} = \frac{1}{4\pi \epsilon_0}\frac{qQ}{r_i^2}\vec{\hat{r_i}}
\end{equation}
\end{shaded}

\subsection{The electric field}

\begin{shaded}
If we have several point charges $q_1,q_1,q_3,...,q_n$ at distances $r_1,r_2,r_3,...,r_n$ from $Q$ the total force on $Q$ is. 
\begin{equation}
\vec{F} = Q\vec{E}
\end{equation}
where 
\begin{equation}
\vec{E}(r) \equiv \frac{1}{4\pi \epsilon_0} \sum\limits_{i=1}^n \frac{q_i}{r_i}\vec{\hat{r_i}}
\end{equation}
\end{shaded}

if we instead have a distribution of charges that we can approximate with a continuous distribution we get  

\begin{shaded}
\begin{equation}
\vec{E}(r) = \frac{1}{4\pi\epsilon_0} \int_V \frac{\vec{\hat{r}}}{r^2} dq
\end{equation}
for volumes $dq = \rho d\tau$ is used, for surfaces $dq = \sigma da$ and for lines $dq = \lambda dl$.
\end{shaded}

\subsection{Gauss law}

\begin{shaded}
The quantitative statement of gauss law in integral form states that 
\begin{equation}
\oint \vec{E} \cdot da = \frac{1}{\epsilon_0} Q_{enc}
\end{equation}
where $Q_{enc}$ is the total charge contained within the surface. we can then also write gauss law in differential form: 
\begin{equation}
\nabla \cdot \vec{E} = \frac{1}{\epsilon_0}\rho
\end{equation}
where $\rho$ is the charge density. 
\end{shaded}
%\begin{abstract}
%\lipsum[10]
%\end{abstract}


\end{document}